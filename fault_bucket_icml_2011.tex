\documentclass{article}

\usepackage{graphicx}
\usepackage{subfigure}
\usepackage{algorithm}
\usepackage{algorithmic}

\usepackage[draft]{hyperref}
\newcommand{\theHalgorithm}{\arabic{algorithm}}

% Employ this version of the ``usepackage'' statement after the paper has
% been accepted, when creating the final version.  This will set the
% note in the first column to ``Appearing in''
% \usepackage[accepted]{icml2011}

\usepackage{icml2011}

\usepackage{xcolor}
\usepackage[american]{babel}
\usepackage{auto-pst-pdf}
\usepackage{microtype}
\usepackage{psfrag}
\usepackage{amsmath}
\usepackage{amssymb}
%\usepackage[T1]{fontenc}
\usepackage{nicefrac}
\usepackage{booktabs}

\usepackage{natbib}

\newcommand{\psff}[1]{\input{#1.tex}\includegraphics{#1.eps}}
\newcommand{\R}{\ensuremath{\mathbb{R}}}
\newcommand{\deq}{\ensuremath{\triangleq}}
\newcommand{\given}{\ensuremath{\mid}}
\newcommand{\cm}[1]{\ensuremath{\mathcal{#1}}}
\newcommand{\bm}[1]{\ensuremath{\mathbf{#1}}}
\newcommand{\data}{\ensuremath{\cm{D}}}
\newcommand{\inv}{\ensuremath{^{-1}}}
\newcommand{\acro}[1]{\textsc{#1}}

% Mike's definitions
\newcommand{\dd}[2]{\delta(#1-#2))}
\newcommand{\vect}[1]{\bm{#1}}
\newcommand{\vz}{\vect{z}}
\newcommand{\vx}{\vect{x}}
\newcommand{\vf}{\vect{f}}
\newcommand{\vs}{\vect{\sigma}}
\newcommand{\amean}[2]{\tilde{{m}}(#1 \given #2 )}
\newcommand{\acov}[2]{\tilde{{C}}(#1 \given #2 )}
\newcommand{\p}[2]{p(#1 \given #2 )}
\newcommand{\fPr}{p}
\newcommand{\Prob}[2]{\fPr(#1 \given #2 )}
\newcommand{\ps}[2]{p(#1\vert#2)}
\newcommand{\mean}[2]{{m}(#1 \given #2 )}
\newcommand{\cov}[2]{{C}(#1 \given #2 )}
\newcommand{\N}[3]{\cm{N}( #1;#2,#3 )}
\newcommand{\st}{_{\star}}
\newcommand{\tr}{\ensuremath{\mathsf{T}}}
\newcommand{\defequal}{\triangleq}

\DeclareMathOperator{\fault}{fault}
\DeclareMathOperator{\diag}{diag}
\DeclareMathOperator{\chol}{chol}

\icmltitlerunning{Fault Bucket for Time-Series Prediction}

\nonstopmode

\begin{document}

\twocolumn[ 

  \icmltitle{A ``Fault Bucket'' for Time-Series Prediction with
    Uncharacterized Faulty Observations}

  \icmlkeywords{Gaussian processes, time-series prediction, fault
    detection, hyperparameter marginalization}

\vskip 0.3in
]

\begin{abstract}
  We provide a proposal for performing both online prediction and
  retrospective inference of signals from observations that are
  potentially rendered less informative than normal due to a faulty
  observation mechanism.  The proposed model uses Gaussian processes
  and a general ``fault bucket'' for \textit{a priori} uncharacterized
  faults, along with an approximate method for marginalizing the
  potential faultiness of all observations. This gives rise to an
  efficient, flexible algorithm. We demonstrate our method's relevance
  to several problems drawn from environmental-monitoring
  applications.
\end{abstract}

\section{Introduction}

We consider the problem of inferring a signal $y\colon \R \to \R$ from
noisy measurements of it.  Typical algorithms for this purpose often
assume that the data is linear, i.i.d., Markovian, or Gaussian. In
some cases, knowledge about the problem can enable the explicit
specification of parameterized nonlinear models. This approach,
however, is problematic for two reasons. First, estimating the model
parameters is often difficult. More crucially, collected observations
in real-world applications are often corrupted in non-trivial ways due
to, for example, a faulty sensing mechanism. Such effects often
inhibit effective environmental monitoring, where data can be
corrupted in unknown ways, rendering the \emph{a priori} construction
of parametric model impossible. The motivating example for this paper
is the related fast-growing field of water-quality monitoring
\citep{wagner2006guidelines}. Despite the enormous importance of
water-quality monitoring to humans, there is a lack of sound
statistical machine-learning solutions for this problem. With this
study, we hope to release some of this data to the public domain
(following the reviewing process in order to satisfy anonymous
reviewing) and to present new machine-learning techniques to meet the
demands of field.

Our proposed method for predicting signals with faulty observation
mechanisms will rely on Gaussian processes (\acro{gp}s) to perform
inference about the underlying latent function.  Previous work has
approached this problem by creating observation models that specify
the anticipated potential fault types \textit{a priori}
\citep{garnettosborne}, but this might be an unreasonable assumption
in highly variable or poorly understood environments.  Here we suggest
the use of a catch-all ``fault bucket,'' which can identify and treat
appropriately data readings suspected of being corrupted in some way.
The result is a method for data-stream prediction that can manage a
wide range of faulty observations without requiring significant
domain-specific knowledge.

The collection of literature on similar topics is vast, under labels
such as fault detection \citep{deFreitas1996, Eciolaza2001,
  Isermann2005, Ding2008}, novelty detection \citep{Markou2003},
anomaly detection \citep{Chandola:2009}, or one-class classification
\citep{Khan2010}. Despite this, most of the techniques are too simple
(e.g. linear or Gaussian) or fail to produce good uncertainty
estimates. Additionally, many problems in anomaly detection and
one-class classification are of a very different nature and,
therefore, the techniques developed there are not immediately
applicable to our domain of interest. Uncertainty estimates are key in
order to provide the user of a system with reliable monitoring
signals. Green-tech areas, including environmental monitoring and
energy-demand prediction, are still far from full
automation. Currently, the most important requirement of such systems
is to provide the user with signals that he or she can use to reach a
decision, making the uncertainty of predictions of the utmost
importance. For this reason, we focus on developing \acro{gp}-based
techniques to build probabilistic nonlinear models of the latent and
fault processes and thereby deliver reliable reports.  In addition to
providing posterior probabilities of observation faultiness, we are
able to perform effective prediction for the latent process even in
the presence of faults.

GPs have been used previously for fault detection
\citep{Eciolaza2001}. However, the nature of the data and setup in
that domain is different from ours, forcing us to develop new
\acro{gp}-based techniques for detection and prediction in the
presence of faults.

\section{Gaussian Processes}

Gaussian processes provide a simple, flexible framework for performing
Bayesian inference about functions \citep{gpml}.  A Gaussian process
is a distribution on the functions $y\colon \cm{X} \to \R$ (on an
arbitrary domain $\cm{X}$) with the property that the distribution of
the function values at a finite subset of points $F \subseteq \cm{X}$
are multivariate Gaussian distributed.

A Gaussian process is completely defined by its first two moments: a
mean function $\mu\colon \cm{X} \to \R$ and a symmetric positive
semidefinite covariance function $K\colon \cm{X} \times \cm{X} \to
\R$.  The mean function describes the overall trend of the function
and is typically set to a constant for convenience.  The covariance
function describes how function values are correlated as a function of
their locations in the domain, thereby encapsulating information about
the overall shape and behavior of the signal.  Many covariance
functions are available to model a wide variety of anticipated
signals.

Suppose we have chosen a Gaussian process prior distribution on the
function $y\colon \cm{X} \to \R$, and a set of input points $\bm{x}$,
the prior distribution on $\bm{y} \deq y(\bm{x})$ is
\begin{equation*}
 p(\bm{y} \given \bm{x}, \theta)
 =
 \cm{N}
 \bigl(
   \bm{y};
   \mu(\bm{x}; \theta),
   K(\bm{x}, \bm{x}; \theta)
 \bigr),
\end{equation*}
where $K(\bm{x}, \bm{x}; \theta)$ is the Gram matrix of the points
$\bm{x}$, and $\theta$ is a vector containing any parameters required
of $\mu$ and $K$, which constitute hyperparameters of the model.

Exact measurements of the latent function are typically not available;
however, we may combine the Gaussian process distribution on the
latent function with an observation model that takes into account
potential noise in our measurements.  Let $z(x)$ represent the
realization of an observation of the signal at $x$ and $y(x)$
represent the value of the unknown true latent signal at that point.
When the observation mechanism is not expected to experience faults,
the usual noise model used is
\begin{equation}\label{iidnoise}
 p(z \given y, x, \sigma_n^2)
 \deq
 \cm{N}(z; y, \sigma_n^2),
\end{equation}
which represents additive i.i.d.\space Gaussian observation noise with
variance $\sigma_n^2$. Note that this model is inappropriate when
sensors can experience faults, which complicate the relationship
between $z$ and $y$.

With the observation model above, given a set of observations
$
 \data
 \deq
 \bigl\lbrace
   \bigl( x, z(x) \bigr)
 \bigr\rbrace
 \deq
 ( \bm{x}, \bm{z} )
$,
the posterior distribution of $y\st \deq y(x\st)$ given these data is
\begin{equation*}
 p(y\st \given \data, \theta)
 =
 \cm{N}
 \bigl(
   y\st;
   \mean{y\st}{\data,\theta},
   \cov{y\st}{\data,\theta}
 \bigr),
\end{equation*}
where the posterior mean and covariance are
\begin{align*}
 &
 \mean{y\st}{\data,\theta}
 \deq
 \mu(x\st; \theta)
 +
 {}
 \\
 &
 \mspace{20mu}
 +
 K(x\st, \bm{x}; \theta)
 \bigl(
 K(\bm{x}, \bm{x}; \theta) + \sigma_n^2 \bm{I}
 \bigr)\inv
 \bigl(
   \bm{z} - \mu(\bm{x}; \theta)
 \bigr)
 \\  
 &
 \cov{y\st}{\data,\theta}
 \deq
 K(x\st, x\st; \theta)
 -
 {}
 \\
 &
 \mspace{20mu}
 -
 K(x\st, \bm{x}; \theta)
 \bigl(
   K(\bm{x}, \bm{x}; \theta) + \sigma_n^2 \bm{I}
 \bigr)\inv
 K(\bm{x}, x\st; \theta).
\end{align*}
Notice that the posterior is also a Gaussian process. 

We now make some definitions for the sake of readability. Henceforth,
we assume that our observations $\vz$ have already been scaled by the
subtraction of the prior mean $\mu(\bm{x}; \theta)$. We will also make
use of the covariance matrix shorthand $K_{m,n} \defequal
K(\vx_m,\vx_n)$. Finally, for now, we'll drop the explicit dependence
of our probabilities on the hyperparameters $\theta$ (it will be
implicitly assumed that all quantities are conditioned on knowledge of
them), and will return to them later.

\section{Fault Bucket}
Rather than specifying explicit parameters for every possible fault
type, we propose a single catch-all ``fault bucket'' that can identify
and treat appropriately measurements that are suspected of being
faulty.  The basic idea is to model faulty observations as being
generated from a Gaussian distribution with a very wide variance;
points that are more likely under this model than under the normal
predictive model of the Gaussian process can reasonably be assumed to
be corrupted in some way, assuming we have a good understanding of the
latent process. It is hoped that a very broad class of faults can be
captured in this way.

To formalize this idea, we choose an observation noise distribution to
replace that in \eqref{iidnoise} that models the noise as independent
but not identically distributed with separate variances for the
non-fault and fault cases.
\begin{align*}
 p(z \given y, x, \neg\fault, \sigma_n^2)
 &
 \deq
 \cm{N}(z; y, \sigma_n^2)
 \\
 p(z \given y, x, \fault, \sigma_f^2)
 &
 \deq
 \cm{N}(z; y, \sigma_f^2),
\end{align*}
where $\fault \in \lbrace 0, 1 \rbrace$ is a binary indicator of
whether the observation $z(x)$ was faulty and $\sigma_f > \sigma_n$ is
the variance around the mean of faulty measurements.  The values of
both $\sigma_n$ and $\sigma_f$ form hyperparameters of our model, and
hence are included in $\theta$.

Of course, {\it a priori}, we do not know if any given observation
will be faulty or not.  Unfortunately, managing our uncertainty about
the faultiness of all available observations is a challenging task. If
we have $N$ observations available, then there are $2^N$ possible
combinations of faultiness. For even moderately sized datasets, it is
computationally infeasible to correctly marginalise over the possible
faultinesses of all observations.

Instead, we propose a sequential approach, applicable for ordered data
such as time series. For time series, the predictant $y\st$ typically
lies in the future, such that our older observations are less
pertinent than our newer observations. The intuition behind our
approach is to, at any time step, `merge' the sum over the faultiness
of the most recent observation. This gives rise to an approximation of
our observation of unknown faultiness as an observation of known
variance, lying between between $\sigma_n^2$ and $\sigma_f^2$. The
more likely an observation's faultiness, the closer its assigned
variance will be to the (large) fault variance, and the less relevant
it will be for inference about the latent process. This approximate
observation can then be used to compute future predictions; we need
never consider the full sum over all observations. Nonetheless, this
approximate marginalisation over faultiness is preferable to
heuristics that would designate all observations as either faulty or
not; our method acknowledges the uncertainty that may exist in
faultiness.

More formally, imagine that we have partitioned our observations
$\data_{a,b}$ into a set of old observations $\data_a=(\vx_a,\vz_a)$
and a set of newer observations $\data_b = (\vx_b,\vz_b)$. Define
$\vs_{a}$ as the (unknown) vector of all noise variances at
observations $\vz_{a}$, and similar for $\vs_{b}$. As we have to sum
over all possible values for these vectors, we'll index the values of
$\vs_{a}$ by $i$ (each given by a different combination of
faultinesses over $\data_a$) and the values of $\vs_{b}$ by $j$. We
now define the covariances
\begin{align*}
 V_a^i & \defequal K_{a,a} + \diag \vs_{a}^i\,, \qquad
 V_b^j \defequal K_{a,a} + \diag \vs_{a}^i \\
 V_{a,b}^{i,j} & \defequal K_{\{a,b\},\{a,b\}} + \diag \{\vs_{a}^i,\vs_{b}^j\}\,,
\end{align*}
where $\diag \vs$ is the diagonal matrix with diagonal $\vs$. 

To initialise our algorithm, imagine that $a$ identifies a small set
of data, such that we can readily compute
\begin{equation}
 \hspace{-0.0cm}p(\vz_a)\!=\!\sum_i  \ps{\vz_a}{\vs^i_a}\fPr(\vs^i_a)
\!=\! \sum_i \N{\vz_a}{0}{V_a^i}\fPr(\vs_a^i) \label{eq:likelihood_a}
\end{equation}
(which is the likelihood of our hyperparameters), so
\begin{equation}
\ps{\vs_a}{\data_{a}} 
= \frac{\ps{\vz_a}{\vs_a}\fPr(\vs_a)}{p(\vz_a)} 
= \frac{\N{\vz_a}{0}{V_a} \fPr(\vs_a)}{p(\vz_a)}\label{eq:psa}\,.
\end{equation}
This distribution also specifies the probability of our observations
$\data_a$ being faulty; for a single observation $\data_a$,
$
\Prob{\fault(\data_a)}{\data_{a}} = \Prob{\sigma_a = \sigma_f}{\data_{a}}
$.

If we were to perform predictions for some $y\st$ using $\data_a$
alone, we'd need to evaluate
\begin{align*}
&\p{y\st}{\data_{a}} = \sum_{i} \Prob{\vs^i_{a}}{\data_a} \p{y\st}{\data_a, \vs^{i}_{a}}\nonumber\\
&=\sum_{i} \Prob{\vs^i_{a}}{\data_a} \N{y\st}{\mean{y\st}{\data_a, \vs^{i}_{a}}}{\cov{y\st}{\data_a, \vs^{i}_{a}}}
\end{align*}
the weighted sum of Gaussian predictions made using the different
possible values for $\vs_{a}$.  Now, we'll perform the `merging' that
we discussed earlier, by approximating this sum of Gaussians as a
moment-matched single Gaussian. It's our hope that our predictions for
$y\st$ are not so sensitive to the noise in our observations that all
such Gaussians are dramatically different. Further, the quality of
this approximation will improve over time---if $y\st$ is far removed
from our old data $\data_a$, then our predictions really will not be
very sensitive to $\sigma_a$. As such, we arrive at
\begin{align}
 &\p{y\st}{\data_{a}} \simeq \N{y\st}{\amean{y\st}{\data_a}}{\acov{y\st}{\data_a}}\,,\label{eq:pya}
\end{align}
where we have
\begin{align}
\amean{y\st}{\data_{a}} & \defequal  K_{\star,a} \tilde{V}_a^{-1} \vz_a\label{eq:ameana}\\
\acov{y\st}{\data_{a}}
& \defequal K_{\star,\star} - K_{\star,a}(\tilde{V}_a^{-1}-\tilde{W}_a^{-1})K_{a,\star} \nonumber\\
& \hspace{2.8cm} - \amean{y\st}{\data_{a,b}}^2 \,.\label{eq:acova}
\end{align}
for
\begin{align}
 \tilde{V}_a^{-1}  & \defequal \sum_i \Prob{\vs^i_{a}}{\data_a} (V_a^i)^{-1}\nonumber\\
 \tilde{W}_a^{-1} & \defequal \sum_i \Prob{\vs^i_{a}}{\data_a} (V_a^i)^{-1}\vz_a \vz_a^\tr (V_a^i)^{-1}\,.\label{eq:Wa}
\end{align}
Note that for $\tilde{W}_a$, explicitly computing (unstable) matrix
inverses can be avoided by solving the appropriate linear equations
using Cholesky factors.  For $\tilde{V}_a$, we can rewrite
$(A^{-1}+B^{-1})^{-1} = A (A+B)^{-1} B$. If $i\in\{0,1\}$ (as it would
be if $a$ identified a single observation which could be either faulty
or not),
\begin{align} \label{eq:inverse_trick}
\tilde{V}_a & = V^0_a\bigl(
\Prob{\vs^1_{a}}{\data_a} V^0_a 
+ 
\Prob{\vs^0_{a}}{\data_a} V^1_a
\bigr)^{-1}V^1_a\,.
\end{align}
If $i$ takes more than two values, we can simply iterate using the
same technique. Having used \eqref{eq:inverse_trick} to compute
$\tilde{V}_a$, we can then calculate
\begin{align}
 \tilde{R}_a & \defequal \chol \tilde{V}_a \label{eq:Ra} \\
 \tilde{T}_a & \defequal \chol (\tilde{R}_a)^{-1} \vz_a \label{eq:Ta} \,,
\end{align}
and use them, along with \eqref{eq:Wa}, to efficiently determine
\eqref{eq:ameana} and \eqref{eq:acova}.

Now, these calculations performed, we imagine receiving further data
$\data_b$. We perform predictions now as
\begin{align*}
\p{y\st}{\data_{a,b}} & = \sum_{i} \sum_{j} \p{y\st}{\data_{a,b}, \vs^{i,j}_{a,b}} \Prob{\vs^{i,j}_{a,b}}{\data_{a,b}}\,.
\end{align*}
The full sums here can easily become too large to actually
evaluate. To simplify, we assume that our later observations are
independent of the noise in our earlier observations. To be more
precise, we approximate as
\begin{align} \label{eq:approx}
%\p{\vs^j_{b},\data_b}{\vs^i_{a},\data_a} & \simeq \p{\vs^j_{b},\data_b}{\data_a} \text{giving}\\
\Prob{\vs^{i,j}_{a,b}}{\data_{a,b}} & \simeq \Prob{\vs^i_{a}}{\data_a}\,\Prob{\vs^j_{b}}{\data_{a,b}}\,,
\end{align}
and
\begin{align}
\hspace{-0.2cm}\p{y\st}{\data_{a,b}} & \simeq \!\sum_{j} \Prob{\vs^j_{b}}{\data_{a,b}}\!\sum_{i} \Prob{\vs^i_{a}}{\data_a} \p{y\st}{\data_{a,b}, \vs^{i,j}_{a,b}}\nonumber\\
& =\!\sum_{j} \Prob{\vs^j_{b}}{\data_{a,b}}\!\sum_{i} \Prob{\vs^i_{a}}{\data_a}\nonumber\\& \hspace{-0.8cm} \N{y\st}{\mean{y\st}{\data_{a,b}, \vs^{i,j}_{a,b}}}{\cov{y\st}{\data_{a,b}, \vs^{i,j}_{a,b}}}\,.\label{eq:sum_o_Gaussians}
\end{align}
Before trying to manage these sums, let's determine
$\Prob{\vs_b}{\data_{a,b}}$. As before, this distribution gives us the
probability of observations $\data_b$ being faulty. For example, if we
have only a single observation $\data_b$, $
\Prob{\fault(\data_b)}{\data_{a,b}} = \Prob{\sigma_b =
  \sigma_f}{\data_{a,b}} $.

 Moving on, we define
\begin{align}
\amean{\vz_b}{\data_a} & \defequal  
K_{b,a} \tilde{V}_a^{-1} \vz_a \label{eq;ameanba}
\\
\acov{\vz_b}{\data_{a},\vs_b}
& \defequal V_b - K_{b,a}(\tilde{V}_a^{-1}-\tilde{W}_a^{-1})K_{a,b} \nonumber\\
& \hspace{2.8cm} - \amean{\vz_b}{\data_{a,b}}^2 \,. \label{eq;acovba}
\end{align}
where both $\tilde{V}_a$ (or its Cholesky factor) and
$\tilde{W}_a^{-1}$ were computed previously. By using
\eqref{eq:approx} and again approximating a sum of Gaussians as a
single Gaussian,
\begin{align}
\Prob{\vs_b}{\data_{a,b}} & = \frac{\sum_i \p{\vz_b}{\data_a,\vs^i_{a,b}}p(\vz_a,\vs^i_{a,b})}{p(\vz_{a,b})}\nonumber\\
& \simeq \frac{\sum_i \ps{\vz_b}{\data_a,\vs^i_{a,b}}\fPr(\vs_a^i\mid{\data_a})\fPr(\vs_{b})}{\ps{\vz_{b}}{\data_a}}\nonumber\\
& \simeq \frac{\N{\vz_b}{\amean{\vz_b}{\data_a}}{\acov{\vz_b}{\data_a, \Sigma_b}} \fPr(\vs_b)}{\p{\vz_{b}}{\data_a}}\,,\label{eq:psb}
\end{align}
where we have
\begin{align}
&\p{\vz_{b}}{\data_a}\nonumber\\
& = \sum_i \sum_j \p{\vz_b}{\data_a,\vs^i_{a,b}}\Prob{\vs^{i,j}_{a,b}}{\data_a}\nonumber\\
& \simeq \sum_i \sum_j \p{\vz_b}{\data_a,\vs^i_{a,b}}\Prob{\vs_a^i}{\data_a}\fPr(\vs_{b}^j)\nonumber\\
& \simeq \sum_j \N{\vz_b}{\amean{\vz_b}{\data_a}}{\acov{\vz_b}{\data_a, \vs_b^j}} \fPr(\vs_b^j)\,.\label{eq:likelihood_b}
\end{align}
Note that the product of \eqref{eq:likelihood_b} and
\eqref{eq:likelihood_a} gives the likelihood of our hyperparameters.

Now, returning to \eqref{eq:sum_o_Gaussians}, we will once again
approximate a sum of Gaussians as a moment-matched single
Gaussian. Our goal here is to re-use our previously evaluated sums
over $i$ to resolve future sums over $i$. As we gain more data, the
faultiness of very old data becomes less important. We arrive at
\begin{align}
\p{y\st}{\data_{a,b}} & \simeq \N{y\st}{\amean{y\st}{\data_{a,b}}}{\acov{y\st}{\data_{a,b}}}\,,\label{eq:pyab}
\end{align}
where we have
\begin{align}
\amean{y\st}{\data_{a,b}} & \defequal  K_{\star,a,b} \tilde{V}_{a,b}^{-1} \vz_{a,b}\label{eq:ameanab}\\
\acov{y\st}{\data_{a}}
& \defequal K_{\star,\star} - K_{\star,a}(\tilde{V}_{a,b}^{-1}-\tilde{W}_{a,b}^{-1})K_{a,\star} \nonumber\\
& \hspace{2.8cm} - \amean{y\st}{\data_{a,b}}^2 \,.\label{eq:acovab}
\end{align}
for
\begin{align*}
 & \tilde{V}^{-1}_{a,b} \defequal 
\sum_{j} \Prob{\vs^j_{b}}{\data_{a,b}}\sum_i \Prob{\vs^i_{a}}{\data_{a}} (V_{a,b}^{i,j})^{-1}
\end{align*}
\begin{align*}
 & \tilde{W}^{-1}_{a,b} \defequal \nonumber\\ 
& \sum_{j} \Prob{\vs^j_{b}}{\data_{a,b}}\sum_i \Prob{\vs^i_{a}}{\data_{a}} (V_{a,b}^{i,j})^{-1}\vz_{a,b}^{\phantom{\tr}} \vz_{a,b}^\tr (V_{a,b}^{i,j})^{-1}\,.
\end{align*}
Now, using the inversion by partitioning formula \citep[Section 2.7]
{NumericalRecipes},
\begin{align*}
\hspace{-0.2cm}&({V}^{i,j}_{a,b})^{-1} = \\
&\begin{bmatrix}
 S^{i,j}_a &\hspace{-0.1cm} -S^{i,j}_a K_{a,b} (V^j_b)^{-1} \\
 - (V^j_b)^{-1} K_{b,a} S^{i,j}_a &\hspace{-0.1cm} (V^j_b)^{-1}\!+\!(V^j_b)^{-1} K_{b,a} S^{i,j}_a K_{a,b} (V^j_b)^{-1} 
\end{bmatrix}
\end{align*}
where
$
S^{i,j}_a \defequal (V^i_a -K_{a,b} (V^j_b)^{-1}K_{b,a})^{-1}\,.
$

Note that $({V}^{i,j}_{a,b})^{-1}$ is affine in $S^{i,j}_a$, so that where
\begin{equation}\label{eq:V_a_big}
 V_a \gg K_{a,b} V_b^{-1} K_{b,a}\,,
\end{equation}
$({V}^{i,j}_{a,b})^{-1}$ is effectively affine in
$(V^i_a)^{-1}$. \eqref{eq:V_a_big} is true if, given $\data_b$, you
are unable to accurately predict $\data_a$. This might be the case if
$\data_a$ represents a lot of information relative to $\data_b$ (if,
for example, $\data_a$ is our entire history of observations where
$\data_b$ is simply the most recent observation), or if $\data_b$ and
$\data_a$ are simply not particularly well-correlated. On this basis,
\eqref{eq:V_a_big} seems reasonable for our application. Defining the
affine $f: (V^i_a)^{-1} \mapsto ({V}^{i,j}_{a,b})^{-1}$, then, and as
$\sum_i \Prob{\vs^i_{a}}{\data_{a}} = 1$,
$$
\sum_i \Prob{\vs^i_{a}}{\data_{a}} f\bigl((V^i_a)^{-1} \bigr) \simeq f\Bigl(\sum_i \Prob{\vs^i_{a}}{\data_{a}}(V^i_a)^{-1} \Bigr)
$$
and so, for
\begin{align*}
\hat{V}^{j}_{a,b} & \defequal
\begin{bmatrix}
 \tilde{V}_a & K_{a,b}\\
 K_{b,a} & V^j_b
\end{bmatrix}
\,,
\quad
 \tilde{V}^{-1}_{a,b} \simeq \sum_j \Prob{\vs^j_{b}}{\data_{a,b}} (\hat{V}^{j}_{a,b})^{-1}
\nonumber\\
 \tilde{V}_{a,b} & \simeq
\begin{bmatrix}
 \tilde{V}_a & K_{a,b}\\
 K_{b,a} & \tilde{V}_{b|a} + K_{b,a} \tilde{V}_a^{-1} K_{a,b}
\end{bmatrix}
\end{align*}
where
\begin{equation*}
 \tilde{V}^{-1}_{b|a} \defequal \sum_j \Prob{\vs^j_{b}}{\data_{a,b}} (V^j_b -K_{b,a} \tilde{V}_a^{-1}K_{a,b})^{-1}
\end{equation*}
which, if $b$ identifies a single observation and $j\in\{0,1\}$, can
be computed using the same trick as in \eqref{eq:inverse_trick}. Note
that the lower right hand element of $\tilde{V}_{a,b}$ defines the
noise variance to be associated with observations $\data_b$. As
before, we determine our predictions \eqref{eq:pyab} by solving linear
equations using the quantities
\begin{align}
 \tilde{R}_{a,b} & \defequal \chol \tilde{V}_{a,b} \label{eq:Rab} \\
 \tilde{T}_{a,b} & \defequal \chol (\tilde{R}_{a,b})^{-1} \vz_a \label{eq:Tab} \,,
\end{align}
both of which can be efficiently determined \citep[Appendix
  B]{osbornebayesian} using the evaluated $\tilde{R}_a$ and
$\tilde{T}_a$.

We now turn to $\tilde{W}_{a,b}^{-1}$. Unfortunately, even if
\eqref{eq:V_a_big} were true, $\tilde{W}_{a,b}^{-1}$ is quadratic in
$(V^i_a)^{-1}$. We will nonetheless assume that $\tilde{W}_{a,b}^{-1}$
is affine in $(V^i_a)^{-1}$. The quality of our approximation for
$\tilde{W}_{a,b}^{-1}$ is much less critical than for
$\tilde{V}^{-1}_{a,b}$, as the former only influences the variance of
our predictions for the current predictant; any flaws in the
approximation will not be propagated forward, multiplying in
influence. Further, of course, where one probability dominates,
$\Prob{\vs^i_{a}}{\data_{a}}\gg \Prob{\vs^{i'}_{a}}{\data_{a}} \forall
i' \neq i$,, the approximation is valid. With our approximation,
\begin{align}
\tilde{W}^{-1}_{a,b} \defequal
& \sum_{j} \Prob{\vs^j_{b}}{\data_{a,b}} (\hat{V}_{a,b}^{j})^{-1}\vz_{a,b}^{\phantom{\tr}} \vz_{a,b}^\tr (\hat{V}_{a,b}^{i,j})^{-1}\,.\label{eq:Wab}
\end{align}
and can solve for $K_{\star,(a,b)}\tilde{W}^{-1}_{a,b}K_{(a,b),\star}$ by efficiently updating using the previously computed quantity $\tilde{T}_{a}$.

%% \begin{figure*}
%%  \begin{centering}
%%  \psff{faultpredictions}
%%  \caption{10-step lookahead Gaussian-process prediction of the river
%%    level using a standard regression model incorporating
%%    fault-bucket Gaussian observation noise.  The prior mean function
%%    was the zero function, the prior covariance function was a
%%    Mat\'{e}rn covariance with $\nu = \nicefrac{5}{2}$, and the
%%    observation model was the fault-bucket noise model.  The blue
%%    line shows posterior mean prediction; light blue bands represent
%%    pointwise $\pm 2$ posterior standard deviations bounds.
%%    Observations are plotted in a color that is indicative of their
%%    inferred fault probabilities; the more red an observation is, the
%%    higher the posterior probability of its being a fault.}
%%  \end{centering}
%%  \label{faultpredictions}
%% \end{figure*}

%% \begin{algorithm}[tb]
%%    \caption{Fault Bucket}
%% \label{alg:fault_bucket}
%% \begin{algorithmic}
%%    \STATE {\bfseries input:} data $(\vx,\vz)$, lookahead $\delta$
%% \STATE $x_a \leftarrow x_1,\, z_a \leftarrow z_1,
%% \,x\st \leftarrow x_{1+\delta}$
%% \\
%% \STATE {\bfseries return:} $p(\vz_a) \leftarrow$\eqref{eq:likelihood_a}
%% \FOR {$i = 0,1$}
%% \STATE {\bfseries return:} $\Prob{\vs^i_a}{\data_{a}} \leftarrow$ \eqref{eq:psa}\\
%% \ENDFOR
%%  \STATE $\tilde{R}_a \leftarrow$ \eqref{eq:Ra}, $\tilde{T}_a\leftarrow$ \eqref{eq:Ta}, \, $\tilde{W}^{-1}_a \leftarrow$ \eqref{eq:Wa}\\
%%  \STATE $\amean{y\st}{\data_a} \leftarrow$ \eqref{eq:ameana}, $\acov{y\st}{\data_a} \leftarrow$ \eqref{eq:acova}\\
%% \STATE {\bfseries return:} $\p{y\st}{\data_{a}} \leftarrow$ \eqref{eq:pya}\\
%% \FOR {$t = 2,\ldots$}
%% \STATE $x_b \leftarrow x_t, z_b \leftarrow z_t, x\st \leftarrow x_{t+\delta}$\\
%% \STATE $\amean{\vz_b}{\data_a} \leftarrow$ \eqref{eq;ameanba}, $\acov{\vz_b}{\data_a} \leftarrow$ \eqref{eq;acovba}\\
%% \STATE $\p{\vz_b}{\data_a} \leftarrow$ \eqref{eq:likelihood_b}\\
%% \FOR {$j = 0,1$}
%% \STATE {\bfseries return:} $\Prob{\vs^j_b}{\data_{a,b}} \leftarrow$ \eqref{eq:psb}\\
%% \ENDFOR
%% \STATE $\tilde{R}_{a,b} \leftarrow$ \eqref{eq:Rab}, $\tilde{T}_{a,b}\leftarrow$ \eqref{eq:Tab}, \, $\tilde{W}^{-1}_{a,b} \leftarrow$ \eqref{eq:Wab}\\
%%  \STATE $\amean{y\st}{\data_{a,b}} \leftarrow$ \eqref{eq:ameanab}, $\acov{y\st}{\data_{a,b}} \leftarrow$ \eqref{eq:acovab}\\
%% \STATE {\bfseries return:} $\p{y\st}{\data_{a,b}}\leftarrow$ \eqref{eq:pyab}\\
%% \STATE {\bfseries return:} $p(\vz_a) \leftarrow p(\vz_a)\p{\vz_b}{\data_a}$
%% \STATE $x_a \leftarrow \{x_a,x_b\},\, z_a \leftarrow \{z_a,z_b\}$\\
%% \STATE $\tilde{R}_a \leftarrow \tilde{R}_{a,b},\, \tilde{T}_a\leftarrow \tilde{T}_{a,b}$\\
%% \ENDFOR 
%% \end{algorithmic}
%% \end{algorithm}

%% An outline of our approach is depicted in Algorithm \ref{alg:fault_bucket}.

\subsection{Discussion}

Firstly, we return to the management of our hyperparameters
$\theta$. Unfortunately, analytically marginalising $\theta$ is
impossible. Most of the hyperparameters of our model can be set by
optimising their likelihood on a large training set, giving a
likelihood close to a delta function. This is not true of the
hyperparameters $\sigma_n$ and $\sigma_f$, due to exactly the same
problematic sums discussed earlier. Instead, we marginalise these
hyperparameters online using Bayesian Monte Carlo \citep[Chapter
  7]{osbornebayesian}, taking a fixed set of samples in their values,
and using the hyperparameter likelihoods (computed in
\eqref{eq:likelihood_a} and \eqref{eq:likelihood_b}) to construct
weights over them. Essentially, Algorithm \ref{alg:fault_bucket} is
run in parallel for each sample, and the predictions from each
combined in a weighted mixture. Note that we can use a similar
procedure \citep{garnettosborne} to determine the full posterior for
$\sigma_n$ and $\sigma_f$.  It would be desirable to use non-fixed
samples, but, unfortunately, this would require Algorithm
\ref{alg:fault_bucket} to be re-run from scratch each time a sample is
moved.

The proposal in Algorithm \ref{alg:fault_bucket} can be extended in
several ways. Firstly, we may wish to sum over a number of possible
variances greater than two---useful if observations are prone to
faultiness in more than one mode.  Note that if, instead of summing
over a small number of known variances, we wish to marginalise with
respect to a density over noise variance, we simply replace the sums
over $i$ and $j$ with appropriate integrals. Obvously this will only
be analytically possible if the posteriors for $\sigma_a$ take
appropriate, simple forms.  These extensions would allow our algorithm
to tackle the general problem of heteroskedasticity.

Our proposed algorithm steps through our data one at a time, so that
$\data_b$ always contains only a single observation. However, it would
be possible in general to step in larger chunks, evaluating larger
sums. While more computationally demanding, this might be expected to
improve results. It would also allow us to consider non-diagonal noise
contributions.

We have so far not specified our prior for faultiness (as expressed by
$p(\sigma_a)$ and $p(\sigma_b)$). Within this paper, we consider
exclusively a time-independent probability of faultiness. However,
within the framework afforded by our approximation \eqref{eq:approx},
we are free to consider noise variances that are expected to change
over time.

In some contexts it might be useful to perform inference about the
fault contribution, rather than the signal of interest.  This task is
trivial; we merely switch the roles of the fault and non-fault
contributions.  To make a prediction about a potential faulty signal
at $x$, we follow exactly the same procedure as above, substituting
$\sigma_n^2$ for $\sigma_f^2$.

\begin{figure*}
  \centering
  \psff{bias}
  \psff{kf_bias}
  \psff{dynamics}
  \psff{kf_dynamics}
  \label{test}
\end{figure*}

\begin{figure*}
  \centering
  \psff{painting}
  \psff{fishkiller}
  \label{test}
\end{figure*}

Further to our provision of the posterior probability of an
observation's faultiness, it might be necessary to make a hard
decision on this quantity. This would be necessary, for example, if a
system had correctional or responsive actions that it could take when
such an event occurred.  Fortunately, we can address this problem
using simple Bayesian decision theory.

\section{Results}
We test the effectiveness of the fault bucket on several time-series
datasets that are indicative of problems found in environmental
monitoring. In particular, we test on water level readings; such data
is often characterized by complex dynamics and will thus provide a
good indicator of performance in real-world tasks. We aim to improve
upon the simple, human-supervised approaches to fault detection used
in this field \citep{wagner2006guidelines}. For a quantitative
assessment, we use three semi-synthetic datasets where a typical fault
has been injected into clean sensor data. We then analyze qualitative
performance on two real data sets with actual faults. All measurements
are given in meters, with samples spaced in increments of
approximately 30 minutes.

We return to the water-level signal with the painting artifacts
plotted in Figure \ref{hgriver}.  To test the method described above,
we performed 10-step lookahead online prediction on the river-level
function using a zero-mean Gaussian process prior distribution and the
fault-bucket observation model.  The covariance function was chosen to
be a Mat\'{e}rn covariance with parameter $\nu = \nicefrac{5}{2}$
\citep{gpml}.  The hyperparameters of the non-fault model (the
characteristic input and output scales and the noise variance
$\sigma_n^2$) were learned offline via maximum-likelihood--II
estimation on a disjoint segment of uncorrupted data.  The fault
variance $\sigma_f^2$ was set to $0.2$ and the fault prior $\alpha$
was chosen to be $10\%$.

Figure \ref{faultpredictions} shows the results.  The predictions
follow the true signal much better than the model that generated the
results in Figure \ref{normalpredictions}.  Additionally, the model
detected the faulty observations with reasonable accuracy, despite the
very small amount of prior knowledge about faults that the
fault-bucket model incorporates beyond ``faulty observations are
unpredictable.''

\subsection{Synthetic Bias Fault}
Our first synthetic example concerns a simple sensor error where
measurements are temporarily adjusted by a constant offset, but
otherwise remain accurate. This could happen if the sensor undergoes
physical trauma which results in a loss of calibration.

\subsection{Synthetic Anomaly}
In this dataset, the water level rises quickly, but smoothly, before
returning back to normal. This would be indicative of a genuine
environmental event such as a flash flood.

\subsection{Painting}
Painting is an error that occurs when ice builds on a sensor obscuring
some of the readings. It is characterized by frequent sensor spikes
interlaced with the original, and still accurate signal.

\subsection{Fishkiller}
Our final dataset, which we dub ``fishkiller'', comes from a sensor
near a dam on a river in British Columbia, Canada. It contains an
otherwise normal water level reading that is occasionally interrupted
by a short period of rapid oscillation. This occurs when dam operators
open and close the floodgates too quickly. When this happens, the
water level on the other side of the dam experiences a rapid drop
during which time salmon can become trapped on the shores and
stranded, leading to suffocation. Detecting these events is critical
to proper regulation of dams in order to save the lives these fish.

\subsection{Quantitative results}
\begin{table}
  \centering
  \caption{Quantitative comparison of different algorithms on the
    synthetic datasets.  For each dataset, the mean squared error
    (\textsc{mse}), the log likelihood of the true data ($\log p(\bm{y}
    \given \bm{x}, \cm{M})$), and the true-positive and false-positive rates
    of detection for faulty points (\textsc{tpr} and \textsc{fpr}),
    respectively, are shown.  The top half of the table refers to the
    bias dataset; the bottom half to the change-in-dynamics
    dataset. The better of each pair of results is highlighted in
    bold.}
  \label{tbl:results}
  \begin{tabular}{ccccc}
    \toprule
    Method & \scshape{mse} & $\log p(\bm{y} \given \bm{x}, \cm{M})$ & \scshape{tpr} & \scshape{fpr} \\
    \cmidrule{1-1} \cmidrule(l){2-5} 
    \scshape{kf} & 0.894 & $-7.11 \times 10^5$ & 0.053 & 0.073 \\
    \scshape{fb} & \textbf{0.034} & \textbf{-360} & \textbf{0.997} & \textbf{0.011} \\
    \midrule
    \scshape{kf} & 0.360 & $-3.54 \times 10^4$ & \textbf{0.835} & 0.046 \\
    \scshape{fb} & \textbf{0.075} & $\mathbf{-1.30 \times 10^4}$ & 0.798 & \textbf{0.006} \\
    \bottomrule
  \end{tabular}
\end{table}

\section{Conclusions}
We have proposed a novel algorithm, Fault Bucket, for managing time
series data corrupted by faults unknown ahead of time. The chief
theoretical contribution of the paper is a sequential algorithm for
marginalising over the possible faultiness of all observations. This
allows for fast, principled prediction in the presence of unknown
faults.

\bibliography{fault_bucket_icml_2011}
\bibliographystyle{icml2011}

\end{document}


